\chapter{Introduction}

Free text documents, that is, documents comprised of unstructured text like it can be found in this thesis, make up a bulk
of modern day information in the form of papers, articles and other documents. The utility of such documents is self-evident, and
there are millions of them in existence. The question arises of how to quickly and accurately access these documents, and for this
purpose document retrieval systems exist.

Document retrieval systems, or information retrieval (IR) systems, offer the ability to find free-text documents, given some user query.
This query can be in the form of a keyword, or a phrase and usually describes what is being searched for, think of a Google search
query. In fact Google's famous Page-Rank algorithm is a great example of such a document retrieval system for websites.

The aim of the thesis is to present word2doc, an ad-hoc document retrieval system to be used by the Lecture Translator
\citep{mller2016} that is being developed by the Interactive Systems Lab at the KIT, along with a working implementation.
This is motivated through the following use-case in the Lecture Translator: The overarching idea is to automatically highlight
important keywords and phrases in the generated translations of the lecture translator, and to link these keywords and phrases to
the document best describing the keyword or phrase. Thus, a need arises to provide relevant documents for highlighted keywords within
the text. An implementation for word2doc can be found here: \url{https://github.com/jundl77/word2doc}.

Many different kinds of document retrieval systems exist already, with many different approaches. Many rely on simple frequency
statistics while others use neural networks. However, word2doc gains a semantic understanding of the query and the
document, and retrieves documents that contextually match the query. This too has been done to a certain degree, however most
approaches only use neural nets to pre-process certain components, and retrieve results using these pre-calculated
components. Word2doc intrinsically learns the semantic relations between documents, and it is, to my knowledge, the first system to
do so.

Word2doc uses a word frequency pre-filtering followed by a document-embedding based neural network. Using a frequency based
document retrieval system developed by Facebook \citep{drqa} as a baseline, I show that this setup has potential. It outscores
Facebook's system when it comes to measuring accuracies on the documents to retrieve. However, it falls short when it comes to
measuring the quality of a retrieved ranked document set. There are two plausible explanations for this shortfall that I will
discuss in chapter \ref{conclusion}.

In the remaining chapter I cover related works and compare word2doc to what has previously been done. I explain what word2doc does
differently, and where it improves upon other systems. In chapter \ref{theory} I discuss the theoretical foundations of word2doc, and
explain word2doc's architecture in detail. Chapter \ref{sec:data} covers various datasets I used, including the training and evaluation
datasets. Chapter \ref{exp} presents the evaluation techniques I use to measure word2doc's performance, as well as giving some
implementation details and backing up several design decisions with empirical data. In chapter \ref{results} I will present my final
results, and in chapter \ref{conclusion} I will summarize and explore opportunities for future work.


\section{Related Work}

Word2doc falls into the category of ad-hoc information retrieval systems. Ad-hoc information retrieval systems are given a query and
have to quickly and spontaneously choose one (or a few) documents out of a huge collection of documents. In word2doc's case this
is the entire Wikipedia corpus. A common example of an ad-hoc information retrieval system would be any web search engine.

Many different approaches exist that perform information retrieval. Classic approaches use statistical methods to determine the best
match, but given the recent success of word embeddings, there are more and more systems that turn to neural networks and embeddings
to determine the result of a query. I will first present a few traditional approach to IR, and then delve into more relevant systems
that use embeddings.


\subsection{Statistical IR Systems}

The most basic approach involves an inverted lookup index, like it is used in elastic search \citep{elasticsearch}. Elastic
search is a common search engine used for example by Wikipedia or GitHub, and an inverted lookup index is a data structure that allows
for very fast full-text searches. An inverted index consists of a list of all unique words that appear in any document, and for
each word, a list of the documents in which it appears. It provides a fast way to find documents that contain words from the query.

Like elastic search, Facebook's document retriever from DrQA \citep{drqa} also uses an inverted index. However, it improves on elastic
search by additionally using ngrams and a hash function to more precisely index documents as well as the query. This is explained in
detail in section \ref{theory:docret}. I use Facebook's document retriever as a baseline to measure wor2doc's performance.

Furthermore, Ponte \& Croft propose QLM (Query Likelihood Model) \citep{qlm} in which a language model of each document is estimated.
Each document is then ranked by the likelihood of the query matching it, according to the estimated language model. Robertson \&
Zaragoza present BM25 \citep{bm25}, a ranking function based on the probabilistic relevance model \citep{prob-relevance-model}. It uses
TF-IDF to rank matching documents by their relevance for a query. Metzler \& Croft propose a linear model \citep{metzler2005} that
takes proximity between query terms into account through Markov random fields. Lavrenko in \citet{lavrenko2001} and
\citet{lavrenko2004} proposes a pseudo relevance feedback (PRF) based method that performs well at the cost of executing two rounds of
retrieval, instead of one. Terms from the ranked documents retrieved in round one are used to augment the query for round two. The
ranked set from the second round is then presented to the user.

There are many more statistical approaches, notably Latent Semantic Indexing \citep{lsi}, Probabilistic Latent Semantic Indexing
\citep{plsi} and Latent Dirichlet Allocation \citep{lda} that utilize global statistical information of the corpus to calculate a
ranking for a query. On the other hand, embedding techniques such as word2vec \citep{word2vec} or doc2vec \citep{doc2vec} focus
on local context information instead, which I talk about next.


\subsection{IR Systems using Embeddings}

There have been multiple works that use embeddings to improve the performance for IR systems in recent years, mainly due to the
big success of word2vec \citep{word2vec} in NLP. In general, these works can be roughly divided into the following two categories.

Models in the first category reformulate the query using embeddings so that the query more accurately reflect a user’s intent and thus
yields better results. For example, \citet{grbovic2015} map queries into an embeddings space and expand a given query with the K-nearest
neighbors in the embedding space to augment the original query. \citet{roy2016} also use word embeddings and the K-nearest neighbor
algorithm to obtain expansion words for the query. \citet{kuzi2016} expand the query by adding terms from the corpus that are
semantically connected to the query. They do this by computing cosine similarities between embeddings of terms occurring in the
query and embeddings of terms that surround the original term in the corpus, similar to word2vec's Continuous Bag-of-Words (CBOW)
model \citep{word2vec}. \citet{zheng2015} propose a framework to learn weights for individual terms occuring in the query.
Weighting terms allows different words in the query to be differently emphasized. \citet{zamani2016} create two query expansion methods to
estimate query language models based on word embeddings and an embedding-based relevance model. The relevance model was developed by
\citet{lavrenko2001} and it is a statistical model used to capture the contextual relevance of documents in the context of IR.
\citet{roy2016b} use kernel density estimation on the embedding space created by term embeddings to calculate a relevance model.
\citet{zamani2016b} estimate embedding vectors for queries based on the individual embeddings of terms occurring in the
query, something word2doc does as well. \citet{diaz2016} explore the difference between global embeddings and corpus and query
specific embeddings, applied to query expansion. Their results suggest that global embeddings the likes of GloVe vectors
\citep{glove} are underperformed by locally trained embeddings.

Models in the second category try to exploit semantic similarities between terms, queries and documents.
Word2doc falls into this category. For example, \citet{clinchant2013} propose a document representation in which they
non-linearly map word embeddings of words occurring in a document into a higher-dimensional space and then aggregate them into
a document-level representation. \citet{ganguly2015} construct a generalized language model, where the mutual independence between
a pair of words no longer holds. Instead, they use word embeddings to derive the transformation probabilities between words. Their
experimental results on TREC 6-8 ad hoc and Robust tasks show that their model significantly outperforms the standard language
model and LDA-smoothed language model baselines. \citet{zuccon2015} propose a translation language model that captures semantic
relations between words in queries and those in relevant documents by using word embeddings within the well-known translation
language model. A bit differently from these works, \citet{mitra2016} present a documents ranking model named DESM in which a
query document relevance score is computed by aggregating the cosine similarities across all the query-document word pairs.

There are also some models that use deep neural nets like MLPs or CNNs to directly learn embeddings for queries and documents
themselves, however they use user generated data like click-through data. Microsoft's DSSM \citep{huang2013} or Microsoft's
CLSM \citep{shen2014} are examples of those.

Word2doc combines these approaches. It performs two rounds of retrieval and augments the second round with information from the
first round, similar to Lavrenko. The first round of retrieval is done through Facebook's document retriever, a purely statistical
approach, while the second round is performed through a deep neural network. Furthermore, it reformulates the query for the
second retrieval round by creating an embedding for it, and ultimately creates its own document embedding scheme to model
semantic relations between documents and documents, and between documents and queries. It does this through the use of a deep
neural net without the use of user generated data like DSSM or CLSM, and it is, to my knowledge, the first system to do so.

\begin{minipage}{\textwidth}
There are some other approaches that do not fall into the two categories above. Namely, \cite{guo2017} who claims that a
query-document match is not necessarily defined through semantic similarity, but more through relevance. They go on to
present a deep relevance matching model (DRMM) for ad-hoc retrieval. The list of studies mentioned in this section is
non-exhaustive. For a more detailed overview look at \citet{mitra2017}.
\end{minipage}

\clearpage
